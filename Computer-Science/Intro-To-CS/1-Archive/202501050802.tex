\clearpage
\phantomsection
\label{202501050802}
\renewcommand{\notetitle}{Types of Knowledge}

\section*{Note Information}
\begin{itemize}
  \item \textbf{ID:} \texttt{202501050802}
  \item \textbf{Timestamp:} \texttt{\today \ \currenttime}
  \item \textbf{Tags:} \texttt{Computer-Science, Intro-To-CS, Introduction}
  \item \textbf{References:}
    \begin{itemize}
      \item \href{https://ocw.mit.edu/courses/6-100l-introduction-to-cs-and-programming-using-python-fall-2022/}{Introduction to CS and Programming using Python}
      \item Guttag, J., Introduction to Computation and Programming Using Python
    \end{itemize}
\end{itemize}


\section*{Main Content}
\textbf{Main Idea}\\
All knowledge can be thought of as either delarative or imperative.\\

\textbf{Explanation}\\
Declarative knowledge is composed of statements of fact. For example, the square root of a number $x$ is equal to $y$ if $y \cdot y = \sqrt{x}$.
Imperative knowledge includes recipes for deducing information. For example, Heron of Alexandria's algorithm (1) for finding the square root of a number.\\


\section*{Review}
\begin{enumerate}
  \item What are the two types of knowledge? Define them and give examples. 
\end{enumerate}


\section*{Links to Other Notes}
\begin{enumerate}
  \item \hyperref[202501050821]{Algorithms}
\end{enumerate}

\section*{Table of Contents}
\begin{itemize}
  \item \hyperref[toc]{TOC}
\end{itemize}



