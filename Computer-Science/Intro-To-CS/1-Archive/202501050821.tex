\clearpage
\phantomsection
\label{202501050821}
\renewcommand{\notetitle}{Algorithms}


\section*{Note Information}
\begin{itemize}
  \item \textbf{ID:} \texttt{202501050821}
  \item \textbf{Timestamp:} \texttt{\today \ \currenttime}
  \item \textbf{Tags:} \texttt{Computer-Science, Intro-To-CS, Introduction}
  \item \textbf{References:}
    \begin{itemize}
      \item \href{https://ocw.mit.edu/courses/6-100l-introduction-to-cs-and-programming-using-python-fall-2022/}{Introduction to CS and Programming using Python}
      \item Guttag, J., Introduction to Computation and Programming Using Python
    \end{itemize}
\end{itemize}


\section*{Main Content}
\textbf{Main Idea}\\
An algorithm is a sequence of simple steps, together with a flow of control that specifies when to execute each step.\\

\textbf{Explanation}\\
For example, Heron of Alexandria's algorithm for computing the square root of a number $x$ is described as follows:
\begin{enumerate}
  \item Start with a guess $g$.
  \item If $ g \cdot g $ is close enough to $x$, stop and say that $g$ is the answer.
  \item Otherwise create a new guess by averaging $g$ and $x/g$.
  \item Using this new guess, which we again call $g$, repeat the process until $g \cdot g$ is close enough to $x$.
\end{enumerate}

\section*{Review}
\begin{enumerate}
  \item Define the term algorithm and give an example. 
\end{enumerate}


\section*{Links to Other Notes}
\begin{itemize}
  \item \href{}{}
\end{itemize}

\section*{Table of Contents}
\begin{itemize}
  \item \hyperref[toc]{TOC}
\end{itemize}

