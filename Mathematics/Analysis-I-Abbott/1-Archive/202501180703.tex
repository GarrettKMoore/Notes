\clearpage
\phantomsection
\label{202501180703}
\renewcommand{\notetitle}{Initial Definition for R}

\section*{Note Information}
\begin{itemize}
  \item \textbf{ID:} \texttt{202501180703}
  \item \textbf{Timestamp:} \texttt{\today \ \currenttime}
  \item \textbf{Tags:} \texttt{Mathematics, Analysis-I-Abbott, The-Axiom-of-Completeness}
  \item \textbf{References:}
    \begin{itemize}
      \item \href{/home/garrett/Personal/References/Mathematics/Analysis-I/Abbott.pdf}{Abbott, S., Understanding Analysis}
    \end{itemize}
\end{itemize}


\section*{Main Content}
\textbf{Main Idea}\\
$R$ is an ordered field and contains $Q$ as a subfield.\\

\textbf{Explanation}\\
$R$ is a field, meaning that addition and multiplication of real numbers are commutative, associative, and the distributive property holds. $R$ also has an order, meaning the following two properties hold:
\begin{enumerate}
  \item If $x \in R$ and $y \in R$, then one and only one of the statements
  \begin{align*}
    x < y, \hspace{25pt} x = y, \hspace{25pt} y < x
  \end{align*}
  is true.
\item If $x,y,z \in R$, if $x < y$ and $y < z$, then $x < z$. 
\end{enumerate}
Finally, $R$ is a set containing $Q$. The operations of addition and multiplication on $Q$ extend to all of $R$ in such a way that every element of $R$ has an additive inverse and every nonzero element of $R$ has a multiplicative inverse.\\

\section*{Review}
\begin{enumerate}
  \item Define the set of real numbers.
\end{enumerate}


\section*{Links to Other Notes}
\begin{itemize}
  \item \hyperref[]{}
\end{itemize}

\section*{Table of Contents}

\begin{itemize}
  \item \hyperref[toc]{TOC}
\end{itemize}

