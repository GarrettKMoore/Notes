\clearpage
\phantomsection
\label{202501180734}
\renewcommand{\notetitle}{Upper and Lower Bounds}

\section*{Note Information}
\begin{itemize}
  \item \textbf{ID:} \texttt{202501180734}
  \item \textbf{Timestamp:} \texttt{\today \ \currenttime}
  \item \textbf{Tags:} \texttt{Mathematics, Analysis-I-Abbott, The-Axiom-of-Completeness}
  \item \textbf{References:}
    \begin{itemize}
      \item \href{/home/garrett/Personal/References/Mathematics/Analysis-I/Abbott.pdf}{Abbott, S., Understanding Analysis}
    \end{itemize}
\end{itemize}


\section*{Main Content}
\textbf{Main Idea}\\
A set $A \subset R$ is bounded above if there exists a number $b \in R$ such that $a \leq b$ for all $a \in A$. The number $b$ is called an upper bound for $A$. Likewise, the set $A$ is bounded below if there exists a lower bound $l \in R$ such that $l \leq a$ for every $a \in A$. \\

\textbf{Explanation}\\


\section*{Review}
\begin{enumerate}
  \item Define upper and lower bounds.
\end{enumerate}


\section*{Links to Other Notes}
\begin{itemize}
  \item \hyperref[202501180703]{Initial Definition for R}
  \item \hyperref[202501180727]{Axiom of Completeness}
\end{itemize}

\section*{Table of Contents}

\begin{itemize}
  \item \hyperref[toc]{TOC}
\end{itemize}

