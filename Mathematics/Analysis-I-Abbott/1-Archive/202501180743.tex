\clearpage
\phantomsection
\label{202501180743}
\renewcommand{\notetitle}{Supremum and Infimum}

\section*{Note Information}
\begin{itemize}
  \item \textbf{ID:} \texttt{202501180743}
  \item \textbf{Timestamp:} \texttt{\today \ \currenttime}
  \item \textbf{Tags:} \texttt{Mathematics, Analysis-I-Abbott, The-Axiom-of-Completeness}
  \item \textbf{References:}
    \begin{itemize}
      \item \href{/home/garrett/Personal/References/Mathematics/Analysis-I/Abbott.pdf}{Abbott, S., Understanding Analysis}
    \end{itemize}
\end{itemize}


\section*{Main Content}
\textbf{Main Idea}\\
A real number $s$ is the least upper bound for a set $A \subset R$ if it meets the following two criteria:
\begin{enumerate}
  \item $s$ is an upper bound for $A$;
  \item if $b$ is any upper bound for $A$, then $s \leq b$.\\
\end{enumerate}

\textbf{Explanation}\\
The least upper bound is frequently called the supremum of the set $A$, denoted $s = \text{ sup } A$.


\section*{Review}
\begin{enumerate}
  \item Define the supremum of a set.
  \item Define the infimum, or the greatest lower bound, of a set.
  \item Are least upper bounds unique? Explain.
  \item Let
  \begin{align*}
    A = \{ \frac{1}{n}: n \in N \} = \{1, \frac{1}{2}, \frac{1}{3}, ... \}.
  \end{align*}
  What is $\text{ sup } A$ and $\text{ inf } A$?
\end{enumerate}


\section*{Links to Other Notes}
\begin{itemize}
  \item \hyperref[202501180703]{Initial Definition for R}
  \item \hyperref[202501180727]{Axiom of Completeness}
  \item \hyperref[202501180734]{Upper and Lower Bounds}
\end{itemize}

\section*{Table of Contents}

\begin{itemize}
  \item \hyperref[toc]{TOC}
\end{itemize}

