\clearpage
\phantomsection
\label{202501181241}
\renewcommand{\notetitle}{Maximum and Minimum}

\section*{Note Information}
\begin{itemize}
  \item \textbf{ID:} \texttt{202501181241}
  \item \textbf{Timestamp:} \texttt{\today \ \currenttime}
  \item \textbf{Tags:} \texttt{Mathematics, Analysis-I-Abbott, The-Axiom-of-Completeness}
  \item \textbf{References:}
    \begin{itemize}
      \item \href{/home/garrett/Personal/References/Mathematics/Analysis-I/Abbott.pdf}{Abbott, S., Understanding Analysis}
    \end{itemize}
\end{itemize}


\section*{Main Content}
\textbf{Main Idea}\\
A real number $a_0$ is a maximum of the set $A$ if $a_0$ is an element of $A$ and $a_0 \geq a$ for all $a \in A$.\\

\textbf{Explanation}\\
The supremum can exist and not be a maximum, but when a maximum exists, then it is also the supremum. \\


\section*{Review}
\begin{enumerate}
  \item Define maximum.
  \item Define minimum.
  \item Consider the open interval 
    \begin{align*}
      (0,2) = \{ x \in R : 0 < x < 2 \},
    \end{align*}
    and the closed interval
    \begin{align*}
      [0,2] = \{ x \in R : 0 \leq x \leq 2 \}.
    \end{align*}
    What are the maximums of the two sets? What are the supremums?
\end{enumerate}


\section*{Links to Other Notes}
\begin{itemize}
  \item \hyperref[202501180703]{Initial Definition for R}
  \item \hyperref[202501180727]{Axiom of Completeness}
  \item \hyperref[202501180734]{Upper and Lower Bounds}
  \item \hyperref[202501180743]{Supremum and Infimum}
\end{itemize}

\section*{Table of Contents}

\begin{itemize}
  \item \hyperref[toc]{TOC}
\end{itemize}

