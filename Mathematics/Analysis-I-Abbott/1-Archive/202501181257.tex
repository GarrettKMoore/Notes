\clearpage
\phantomsection
\label{202501181257}
\renewcommand{\notetitle}{Q and the Axiom of Completeness}

\section*{Note Information}
\begin{itemize}
  \item \textbf{ID:} \texttt{202501181257}
  \item \textbf{Timestamp:} \texttt{\today \ \currenttime}
  \item \textbf{Tags:} \texttt{Mathematics, Analysis-I-Abbott, The-Axiom-of-Completeness}
  \item \textbf{References:}
    \begin{itemize}
      \item \href{/home/garrett/Personal/References/Mathematics/Analysis-I/Abbott.pdf}{Abbott, S., Understanding Analysis}
    \end{itemize}
\end{itemize}


\section*{Main Content}
\textbf{Main Idea}\\
The Axiom of Completeness is not a valid statement about $Q$.\\

\textbf{Explanation}\\
Consider the set
\begin{align*}
  S = \{r \in Q : r^2 < 2 \}.
\end{align*}
This set is certainly bounded above, however, when we search for the least upper bound, we can always find a smaller supremum. For example, we might try $b = 2$, $b = 3/2$, $b=142/100$, $b = 1415/1000$, and so on. 


\section*{Review}
\begin{enumerate}
  \item Is the Axiom of Completeness a valid statement about $Q$? Explain.
  \item Does the set
    \begin{align*}
      S = \{ r \in Q : r^2 < 2 \}
    \end{align*}
    have a supremum under $R$?
\end{enumerate}


\section*{Links to Other Notes}
\begin{itemize}
  \item \hyperref[202501180703]{Initial Definition for R}
  \item \hyperref[202501180727]{Axiom of Completeness}
  \item \hyperref[202501180734]{Upper and Lower Bounds}
  \item \hyperref[202501180743]{Supremum and Infimum}
  \item \hyperref[202501181241]{Maximum and Minimum}
\end{itemize}

\section*{Table of Contents}

\begin{itemize}
  \item \hyperref[toc]{TOC}
\end{itemize}

