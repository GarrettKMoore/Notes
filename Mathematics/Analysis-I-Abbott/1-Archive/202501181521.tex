\clearpage
\phantomsection
\label{202501181521}
\renewcommand{\notetitle}{Additional Review}

\section*{Note Information}
\begin{itemize}
  \item \textbf{ID:} \texttt{202501181521}
  \item \textbf{Timestamp:} \texttt{\today \ \currenttime}
  \item \textbf{Tags:} \texttt{Mathematics, Analysis-I-Abbott, The-Axiom-of-Completeness}
  \item \textbf{References:}
    \begin{itemize}
      \item \href{/home/garrett/Personal/References/Mathematics/Analysis-I/Abbott.pdf}{Abbott, S., Understanding Analysis}
    \end{itemize}
\end{itemize}


\section*{Main Content}
\textbf{Questions}\\
\begin{enumerate}
  \item 
    \begin{enumerate}
      \item[(a)] Write a formal definition in the style of \hyperref[202501180743]{Supremum and Infimum} for the infimum or greatest lower bound of a set.
      \item[(b)] Now, state and prove a version of \hyperref[202501181335]{Alternative Phrasing for Supremum} for greatest lower bounds.
    \end{enumerate}
  
  \item Give an example of each of the following, or state that the request is impossible.
    \begin{enumerate}
      \item[(a)] A set $B$ with $\text{ inf } B \geq \text{ sup } B$.
      \item[(b)] A finite set that contains its infimum but not its supremum.
      \item[(c)] A bounded subset $Q$ that contains its supremum but not its infimum.
    \end{enumerate}

  \item 
    \begin{enumerate}
      \item[(a)] Let $A$ be nonempty and bounded below, and define $B = \{b \in R: b \text{ is a lower bound for } A\}$. Show that $\text{ sup } B = \text{ inf } A$.
      \item[(b)] Use (a) to explain why there is no need to assert that greatest lower bounds exist as part of the Axiom of Completeness.
    \end{enumerate}

  \item As in \hyperref[202501181310]{sup(c + A) = c + sup A}, let $A \subset R$ be nonempty and bounded above, and let $c \in R$. This time define the set $cA = \{ca : a \in A\}$.
    \begin{enumerate}
      \item[(a)] If $c \geq 0$, show that $\text{ sup}(cA) = c \text{ sup } A$.
      \item[(b)] Postulate a similar type of statement for $\text{ sup}(cA)$ for the case $c < 0$.
    \end{enumerate}
\end{enumerate}

\textbf{Solutions}\\
\begin{enumerate}
  \item 
    \begin{enumerate}
      \item[(a)] A real number $n$ is the greatest lower bound for a set $A \subset R$ if it meets the following two criteria:
        \begin{enumerate}
          \item[1.] $n$ is a lower bound for $A$;
          \item[2.] if $b$ is any lower bound for $A$, then $b \leq n$. 
        \end{enumerate}
      \item[(b)] Assume $n \in R$ is a lower bound for a set $A \subset R$. Then, $n = \text{ inf } A$ if and only if, for every choice of $\epsilon > 0$, there exists an element $a \in A$ satisfying $a < n + \epsilon$. \\

        \begin{proof}
          Assume $n = \text{ inf } A$ and consider $n + \epsilon$, where $\epsilon > 0$ has been chosen arbitrarily. Because $n < n + \epsilon$, the definition for infimum implies that $n + \epsilon$ is not a lower bound for $A$. Thus, there must be some element $a \in A$ such that $a < n + \epsilon$.\\
          Conversely, assume there exists an element $a \in A$ that satisfies $a < n + \epsilon$. In other words, for any number $b$ that is greater than $n$, $b$ is not a lower bound. Thus, according to the definition, $n$ is the greatest lower bound for $A$. 
        \end{proof}
    \end{enumerate}

  \item 
    \begin{enumerate}
      \item[(a)] Consider $B = \{0\}$; $\text{ sup } B = \text{ inf } B = 0$, thus, $\text{ inf }B \geq \text{ sup } B$.
      \item[(b)] Impossible, finite sets must have both a maximum and minimum, and thus, must contain their infimum and supremum.
      \item[(c)] Consider $B = \{b \in Q : 0 < b \leq 1\}$; $\text{ sup } B = 1 \in B$ and $\text{ inf } B = 0 \not\in B$.
    \end{enumerate}

    \item
      \begin{enumerate}
        \item[(a)] Since every $b \in B$ is a lower bound for $A$, we have $b \leq a$ for all $a \in A$. In particular, $\text{ inf } A$, being the greatest lower bound of $A$, satisfies $b \leq \text{ inf } A$ for all $b \in B$. Thus, $\text{ sup } B \leq \text{ inf } A$, since $\text{ sup } B$ is the least upper bound of $B$. \\
        Conversely, by definition of $\text{ inf } A$, $\text{ inf } A$ is a lower bound for $A$, so $\text{ inf } A \in B$. Since $\text{ sup } B$ is the least upper bound for $B$, it must satisfy $\text{ sup } B \geq \text{ inf } A$. \\
        Therefore, $\text{ sup } B = \text{ inf } A$.
      \item[(b)] The existence of the infimum for a bounded below set $A$ can always be derived from the Axiom of Completeness as follows:
        \begin{itemize}
          \item Define $B$ to be the set of all lower bounds of $A$.
          \item The Axiom of Completeness guarantees that $B$ has a supremum $\text{ sup } B$.
          \item By definition and part (a), $\text{ sup } B = \text{ inf } A$.
        \end{itemize}
        Thus, the existence of greatest lower bounds (infima) is already implicit in the Axiom of Completeness, as every bounded below set can be "reduced" to a problem of finding the supremum of its set of lower bounds.
      \end{enumerate}

    \item 
      \begin{enumerate}
        \item[(a)] Let $s = \text{ sup } A$. We see that $a \leq s$ for all $a \in A$, which implies $ca \leq cs$ for all $a \in A$. Thus, $cs$ is an upper bound for $cA$ and condition (1) of \hyperref[202501180743]{Supremum and Infimum} is verified. For (2), let $b$ be an arbitrary upper bound for $cA$, thus $ca \leq b$ for all $a \in A$. This is equivalent to $a \leq b/c$ for all $a \in A$, from which we conclude that $b/c$ is an upper bound for $A$. Because $s$ is the least upper bound of $A$, $s \leq b/c$, which can be rewritten as $cs \leq b$. This verifies part (2) of \hyperref[202501180743]{Supremum and Infimum}, and we conclude $\text{ sup }(cA) = c \text{ sup } A$. 
        \item[(b)] If $c < 0$, $\text{ sup}(cA) = c \text{ inf } A$.
 
      \end{enumerate}
\end{enumerate}

\section*{Review}
\begin{enumerate}
  \item 
\end{enumerate}


\section*{Links to Other Notes}
\begin{itemize}
  \item \hyperref[202501180703]{Initial Definition for R}
  \item \hyperref[202501180727]{Axiom of Completeness}
  \item \hyperref[202501180734]{Upper and Lower Bounds}
  \item \hyperref[202501180743]{Supremum and Infimum}
  \item \hyperref[202501181241]{Maximum and Minimum}
  \item \hyperref[202501181257]{Q and the Axiom of Completeness}
  \item \hyperref[202501181310]{sup(c + A) = c + sup A}
  \item \hyperref[202501181335]{Alternative Phrasing for Supremum}
\end{itemize}

\section*{Table of Contents}

\begin{itemize}
  \item \hyperref[toc]{TOC}
\end{itemize}

