\clearpage
\phantomsection
\label{202501191115}
\renewcommand{\notetitle}{Nested Interval Property}

\section*{Note Information}
\begin{itemize}
  \item \textbf{ID:} \texttt{202501191115}
  \item \textbf{Timestamp:} \texttt{\today \ \currenttime}
  \item \textbf{Tags:} \texttt{Mathematics, Analysis-I-Abbott, Consequences-of-Completeness}
  \item \textbf{References:}
    \begin{itemize}
      \item \href{/home/garrett/Personal/References/Mathematics/Analysis-I/Abbott.pdf}{Abbott, S., Understanding Analysis}
    \end{itemize}
\end{itemize}


\section*{Main Content}
\textbf{Main Idea}\\
For each $n \in N$, assume we are given a closed interval $I_n= [a_n, b_n] = \{ x \in R : a_n \leq x \leq b_n \}$. Assume also that each $I_n$ contains $I_{n+1}$. Then, the resulting nested sequence of closed intervals
\begin{align*}
  I_1 \supset I_2 \supset I_3 \supset I_4 \supset ...
\end{align*}
has a nonempty intersection; that is, $\bigcap_{n=1}^\infty I_n \neq \emptyset$.\\

\textbf{Explanation}\\
Consider the set 
\begin{align*}
  A = \{ a_n : n \in N \}
\end{align*}
of left-hand endpoints of the intervals. Because the intervals are nested, we see that every $b_n$ serves as an upper bound for $A$. Thus, we are justified in setting
\begin{align*}
  x = \text{ sup } A.
\end{align*}
Now, consider a particular $I_n = [a_n , b_n ]$. Because $x$ is an upper bound for $A$, we have $a_n \leq x$. The fact that each $b_n$ is an upper bound for $A$ and that $x$ is the least upper bound implies $x \leq b_n$.\\
Therefore, we have $a_n \leq x \leq b_n$, which means $x \in I_n$ for every choice of $n \in N$. Hence, $x \in \bigcap_{n=1}^\infty I_n$, and the intersection is not empty.\\


\section*{Review}
\begin{enumerate}
  \item State and prove the Nested Interval Property.
\end{enumerate}


\section*{Links to Other Notes}
\begin{itemize}
  \item \hyperref[202501180703]{Initial Definition for R}
  \item \hyperref[202501180727]{Axiom of Completeness}
  \item \hyperref[202501180734]{Upper and Lower Bounds}
  \item \hyperref[202501180743]{Supremum and Infimum}
  \item \hyperref[202501181241]{Maximum and Minimum}
  \item \hyperref[202501181257]{Q and the Axiom of Completeness}
  \item \hyperref[202501181310]{sup(c + A) = c + sup A}
  \item \hyperref[202501181335]{Alternative Phrasing for Supremum}
\end{itemize}

\section*{Table of Contents}

\begin{itemize}
  \item \hyperref[toc]{TOC}
\end{itemize}

