\clearpage
\phantomsection
\label{202501132004}
\renewcommand{\notetitle}{Rationals are Inadequate}

\section*{Note Information}
\begin{itemize}
  \item \textbf{ID:} \texttt{202501132004}
  \item \textbf{Timestamp:} \texttt{\today \ \currenttime}
  \item \textbf{Tags:} \texttt{Mathematics, Analysis-I, The-Real-and-Complex-Number-System}
  \item \textbf{References:}
    \begin{itemize}
      \item \href{https://ocw.mit.edu/courses/18-100b-analysis-i-fall-2010/}{Analysis I}
      \item \href{/home/garrett/Personal/References/Mathematics/Analysis-I/Rudin.pdf}{Rudin W., Principles of Mathematical Analysis}
    \end{itemize}
\end{itemize}


\section*{Main Content}
\textbf{Main Idea}\\
The rational number system is inadequate for many purposes, both as a field and as an ordered set.\\

\textbf{Explanation}\\
For example, there is no rational $p$ such that $p^2 = 2$.\\

\begin{proof}
  Suppose on the contrary there was a $p$ that satisfied $p^2 = 2$. We could write $p = \frac{m}{n}$, where $m$ and $n$ are integers and coprime. The original expression implies
  \begin{align*}
    (\frac{m}{n})^2 &= 2\\
    \frac{m^2}{n^2} &= 2\\
    m^2 &= 2n^2\\
  \end{align*}
  From this expression, we see that $m^2$ is even, and thus, $m$ is even. Plugging $2k$ in for $m$, it is clear that $m^2$ is divisible by 4. It follows that $2n^2$ is divisible by 4 as well, which implies $n^2$ is even, and thus, $n$ is even. Therefore, our assumption leads to a contradiction that both $m$ and $n$ are even, thus violating the coprime property of $m$ and $n$. Hence, it is impossible for $p$ to be rational. 
\end{proof}

\begin{proof}[Alternative]
 Let $A$ be the set of all positive rationals $p$ such that $p^2 < 2$ and let $B$ consist of all positive rationals $p$ such that $p^2 > 2$. By showing there is no largest element in $A$ and no smallest element in $B$, we effectively partion the set of rational numbers, thus implying there is no rational $p$ that falls outside these two sets, therefore satisfying $p^2 = 2$. To prove that for every $p$ in $A$ we can find a rational $q$ in $A$ such that $p < q$, we associate with each rational $p > 0$ the number 
 \begin{align}
   q &= p - \frac{p^2 - 2}{p + 2} = \frac{2p + 2}{p + 2}.
 \end{align}
 Then
 \begin{align}
   q^2 - 2 &= \frac{2(p^2 - 2)}{(p+2)^2}.
 \end{align}
 If $p$ is in $A$ then $p^2 - 2 < 0$, (1) shows that $q > p$, and (2) shows that $q^2 < 2$. Thus $q$ is in $A$. If $p$ is in $B$ then $p^2 -2 > 0$, (1) shows that $0 < q < p$, and (2) shows that $q^2 > 2$. Thus $q$ is in $B$.\\ 
\end{proof}

\section*{Review}
\begin{enumerate}
  \item Prove that there is no rational $p$ such that $p^2 = 2$ in two different ways.\\
\end{enumerate}


\section*{Links to Other Notes}
\begin{itemize}
  \item \hyperref[202501131947]{Definition of Rational Numbers}
\end{itemize}

\section*{Table of Contents}

\begin{itemize}
  \item \hyperref[toc]{TOC}
\end{itemize}

