\clearpage
\phantomsection
\label{202501141250}
\renewcommand{\notetitle}{Upper Bounds and Lower Bounds}

\section*{Note Information}
\begin{itemize}
  \item \textbf{ID:} \texttt{202501141250}
  \item \textbf{Timestamp:} \texttt{\today \ \currenttime}
  \item \textbf{Tags:} \texttt{Mathematics, Analysis-I, The-Real-and-Complex-Number-Systems}
  \item \textbf{References:}
    \begin{itemize}
      \item \href{https://ocw.mit.edu/courses/18-100b-analysis-i-fall-2010/}{Analysis I}
      \item \href{/home/garrett/Personal/References/Mathematics/Analysis-I/Rudin.pdf}{Rudin W., Principles of Mathematical Analysis}
    \end{itemize}
\end{itemize}


\section*{Main Content}
\textbf{Main Idea}\\
Suppose $S$ is an orderd set, and $E \subset S$. If there exists a $\beta \in S$ such that $x \leq \beta$ for every $x \in E$, we say that $E$ is bounded above, and call $\beta$ an upper bound of $E$. Lower bounds are defined in the same way.\\

\textbf{Explanation}\\


\section*{Review}
\begin{enumerate}
  \item Define upper bound. 
  \item Define lower bound.
\end{enumerate}


\section*{Links to Other Notes}
\begin{itemize}
  \item \hyperref[202501141228]{Order}
  \item \hyperref[202501141241]{Ordered Set}
\end{itemize}

\section*{Table of Contents}

\begin{itemize}
  \item \hyperref[toc]{TOC}
\end{itemize}

