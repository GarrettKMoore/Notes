\clearpage
\phantomsection
\label{202501141546}
\renewcommand{\notetitle}{Supremum and Infimum}

\section*{Note Information}
\begin{itemize}
  \item \textbf{ID:} \texttt{202501141546}
  \item \textbf{Timestamp:} \texttt{\today \ \currenttime}
  \item \textbf{Tags:} \texttt{Mathematics, Analysis-I, The-Real-and-Complex-Number-Systems}
  \item \textbf{References:}
    \begin{itemize}
      \item \href{https://ocw.mit.edu/courses/18-100b-analysis-i-fall-2010/}{Analysis I}
      \item \href{/home/garrett/Personal/References/Mathematics/Analysis-I/Rudin.pdf}{Rudin W., Principles of Mathematical Analysis}
    \end{itemize}
\end{itemize}


\section*{Main Content}
\textbf{Main Idea}\\
Suppose $S$ is an ordered set, $E \subset S$, and $E$ is bounded above. Suppose there exists an $\alpha \in S$ with the following properties:
\begin{itemize}
  \item $\alpha$ is an upper bound of $E$.
  \item If $\gamma < \alpha$ then $\gamma$ is not an upper bound of $E$.  
\end{itemize}
Then $\alpha$ is called the least upper bound of $E$ or the supremum of $E$, and we write
\begin{align*}
  \alpha = \text{ sup } E.
\end{align*}
The greatest lower bound, or infimum, of a set $E$ which is bounded below is defined in the same manner: If $\alpha = \text{ inf } E$, then $\alpha$ is a lower bound of $E$ and no $\beta$ with $\beta > \alpha$ is a lower bound of $E$.\\

\textbf{Explanation}\\
For example, consider the sets $A$ and $B$ from the alternative proof in \hyperref[202501132004]{Rationals are Inadequate}. $A$ and $B$ are subsets of the ordered set $Q$. The set $A$ is bounded above by the members of $B$. Since $B$ contains no smallest member, $A$ has no supremum in $Q$. $B$ is bounded below by the members of $A$. Since $A$ has no largest member, $B$ has no infimum in $Q$.\\


\section*{Review}
\begin{enumerate}
  \item Define supremum.
  \item Define infimum.
  \item If $\alpha = \text{ sup } E$ exists, then must $\alpha$ be a member of $E$? Give an example to justify your answer.
  \item Let $E$ consist of all numbers $1/n$, where $n = 1, 2, 3, ...$. What is $\text{ sup } E$ and $\text{ inf } E$? Are $\text{ sup } E$ and $\text{ inf } E$ members of $E$?
\end{enumerate}


\section*{Links to Other Notes}
\begin{itemize}
  \item \hyperref[202501131947]{Definition of Rational Numbers}
  \item \hyperref[202501132004]{Rationals are Inadequate}
  \item \hyperref[202501141228]{Order}
  \item \hyperref[202501141241]{Ordered Set}
  \item \hyperref[202501141250]{Upper Bounds and Lower Bounds}
\end{itemize}

\section*{Table of Contents}

\begin{itemize}
  \item \hyperref[toc]{TOC}
\end{itemize}

