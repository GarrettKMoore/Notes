\clearpage
\phantomsection
\label{202501141632}
\renewcommand{\notetitle}{Least-Upper-Bound Property}

\section*{Note Information}
\begin{itemize}
  \item \textbf{ID:} \texttt{202501141632}
  \item \textbf{Timestamp:} \texttt{\today \ \currenttime}
  \item \textbf{Tags:} \texttt{Mathematics, Analysis-I, The-Real-and-Complex-Number-Systems}
  \item \textbf{References:}
    \begin{itemize}
      \item \href{https://ocw.mit.edu/courses/18-100b-analysis-i-fall-2010/}{Analysis I}
      \item \href{/home/garrett/Personal/References/Mathematics/Analysis-I/Rudin.pdf}{Rudin W., Principles of Mathematical Analysis}
    \end{itemize}
\end{itemize}


\section*{Main Content}
\textbf{Main Idea}\\
An ordered set $S$ is said to have the least-upper-bound property if the following is true: If $E \subset S$, $E$ is not empty, and $E$ is bounded above, then $\text{ sup } E$ exists in $S$.\\ 

\textbf{Explanation}\\


\section*{Review}
\begin{enumerate}
  \item Define the least-upper-bound property.
  \item Define the greatest-lower bound property.
  \item Does $Q$ have the least-upper-bound property? Explain.
\end{enumerate}


\section*{Links to Other Notes}
\begin{itemize}
  \item \hyperref[202501131947]{Definition of Rational Numbers}
  \item \hyperref[202501132004]{Rationals are Inadequate}
  \item \hyperref[202501141228]{Order}
  \item \hyperref[202501141241]{Ordered Set}
  \item \hyperref[202501141250]{Upper Bounds and Lower Bounds}
  \item \hyperref[202501141546]{Supremum and Infimum}
\end{itemize}

\section*{Table of Contents}

\begin{itemize}
  \item \hyperref[toc]{TOC}
\end{itemize}

