\clearpage
\phantomsection
\label{202501141654}
\renewcommand{\notetitle}{Relation between Sup and Inf}

\section*{Note Information}
\begin{itemize}
  \item \textbf{ID:} \texttt{202501141654}
  \item \textbf{Timestamp:} \texttt{\today \ \currenttime}
  \item \textbf{Tags:} \texttt{Mathematics, Analysis-I, The-Real-and-Complex-Number-Systems}
  \item \textbf{References:}
    \begin{itemize}
      \item \href{https://ocw.mit.edu/courses/18-100b-analysis-i-fall-2010/}{Analysis I}
      \item \href{/home/garrett/Personal/References/Mathematics/Analysis-I/Rudin.pdf}{Rudin W., Principles of Mathematical Analysis}
    \end{itemize}
\end{itemize}


\section*{Main Content}
\textbf{Main Idea}\\
Suppose $S$ is an ordered set with the least-upper-bound property, $B \subset S$, $B$ is not empty, and $B$ is bounded below. Let $L$ be the set of all lower bounds of $B$. Then 
\begin{align*}
  \alpha = \text{ sup } L
\end{align*}
exists in $S$, and $\alpha = \text{ inf } B$. In particular, $\text{ inf } B$ exists in $S$.\\

\textbf{Explanation}\\
Since $B$ is bounded below, $L$ is not empty. Since $L$ consists of exactly those $y \in S$ which satisfy the inequality $y \leq x$ for every $x \in B$, we see that every $x \in B$ is an upper bound of $L$. Thus $L$ is bounded above. Our hypothesis about $S$ implies therefore that $L$ has a supremum in $S$, namely $\alpha$. If $\gamma < \alpha$ then $\gamma$ is not an upper bound of $L$, hence $\gamma \not\in B$. It follows that $\alpha \leq x$ for every $x \in B$. Thus $\alpha \in L$. If $\alpha < \beta$ then $\beta \not \in L$, since $\alpha$ is an upper bound of $L$. We have shown that $\alpha \in L$ but $\beta \not \in L$ if $\beta > \alpha$. In other words $\alpha$ is a lower bound of $B$, but $\beta$ is not if $\beta > \alpha$. This means that $\alpha = \text{ inf } B$.  \\


\section*{Review}
\begin{enumerate}
  \item Prove that if an ordered set has the least-upper-bound property, then it also has the greatest-lower-bound property. 
\end{enumerate}


\section*{Links to Other Notes}
\begin{itemize}
  \item \hyperref[202501131947]{Definition of Rational Numbers}
  \item \hyperref[202501132004]{Rationals are Inadequate}
  \item \hyperref[202501141228]{Order}
  \item \hyperref[202501141241]{Ordered Set}
  \item \hyperref[202501141250]{Upper Bounds and Lower Bounds}
  \item \hyperref[202501141546]{Supremum and Infimum}
  \item \hyperref[202501141632]{Least-Upper-Bound Property}
\end{itemize}

\section*{Table of Contents}

\begin{itemize}
  \item \hyperref[toc]{TOC}
\end{itemize}

