\clearpage
\phantomsection
\label{202501150657}
\renewcommand{\notetitle}{Definition of Field}

\section*{Note Information}
\begin{itemize}
  \item \textbf{ID:} \texttt{202501150657}
  \item \textbf{Timestamp:} \texttt{\today \ \currenttime}
  \item \textbf{Tags:} \texttt{Mathematics, Analysis-I, The-Real-and-Complex-Number-Systems}
  \item \textbf{References:}
    \begin{itemize}
      \item \href{https://ocw.mit.edu/courses/18-100b-analysis-i-fall-2010/}{Analysis I}
      \item \href{/home/garrett/Personal/References/Mathematics/Analysis-I/Rudin.pdf}{Rudin W., Principles of Mathematical Analysis}
    \end{itemize}
\end{itemize}


\section*{Main Content}
\textbf{Main Idea}\\
A field is a set $F$ with two operations, called addition and multiplication, which satisfy the following field axioms (A), (M), and (D):
\begin{itemize}
  \item[(A)] Axioms for addition
    \begin{itemize}
      \item[(A1)] If $x \in F$ and $y \in F$, then their sum $x+y$ is in $F$.
      \item[(A2)] Addition is commutative: $x+y = y+x$ for all $x, y \in F$.
      \item[(A3)] Addition is associative: $(x+y) + z = x + (y+z)$ for all $x,y,z \in F$.
      \item[(A4)] $F$ contains an element 0 such that $0 + x = x$ for every $x \in F$.
      \item[(A5)] To every $x \in F$ corresponds an element $-x \in F$ such that $x + (-x) = 0$.
    \end{itemize}
  \item[(M)] Axioms for multiplication
    \begin{itemize}
      \item[(M1)] If $x \in F$ and $y \in F$, then their product $xy$ is in $F$.
      \item[(M2)] Multiplication is commutative: $xy = yx$ for all $x,y \in F$.
      \item[(M3)] Multiplication is associative: $(xy)z = x(yz)$ for all $x,y,z \in F$.
      \item[(M4)] $F$ contains an element $1 \neq 0$ such that $1x = x$ for every $x \in F$.
      \item[(M5)] If $x \in F$ and $x \neq 0$ then there exists an element $1/x \in F$ such that $x \cdot (1/x) = 1$.
    \end{itemize}
  \item[(D)] The distributive law
    \begin{itemize}
      \item[] $x(y+z) = xy + xz$ holds for all $x,y,z \in F$
    \end{itemize}
\end{itemize}

\textbf{Explanation}\\


\section*{Review}
\begin{enumerate}
  \item Define field.
  \item Is $Q$ a field?
\end{enumerate}


\section*{Links to Other Notes}
\begin{itemize}
  \item \hyperref[202501131947]{Definition of Rational Numbers}
  \item \hyperref[202501132004]{Rationals are Inadequate}
\end{itemize}

\section*{Table of Contents}

\begin{itemize}
  \item \hyperref[toc]{TOC}
\end{itemize}

