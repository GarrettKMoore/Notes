\clearpage
\phantomsection
\label{202501150717}
\renewcommand{\notetitle}{Implications of the Addition Axioms}

\section*{Note Information}
\begin{itemize}
  \item \textbf{ID:} \texttt{202501150717}
  \item \textbf{Timestamp:} \texttt{\today \ \currenttime}
  \item \textbf{Tags:} \texttt{Mathematics, Analysis-I, The-Real-and-Complex-Number-Systems}
  \item \textbf{References:}
    \begin{itemize}
      \item \href{https://ocw.mit.edu/courses/18-100b-analysis-i-fall-2010/}{Analysis I}
      \item \href{/home/garrett/Personal/References/Mathematics/Analysis-I/Rudin.pdf}{Rudin W., Principles of Mathematical Analysis}
    \end{itemize}
\end{itemize}


\section*{Main Content}
\textbf{Main Idea}\\
The axioms for addition imply the following statements.
\begin{itemize}
  \item[(a)] If $x + y = x+z$ then $y=z$.
  \item[(b)] If $x+y = x$ then $y = 0$.
  \item[(c)] If $x+y = 0$ then $y = -x$.
  \item[(d)] $-(-x) = x$.\\
\end{itemize}

\textbf{Explanation}\\
\begin{proof}
  If $x + y = x + z$, the axioms $(A)$ give
  \begin{align*}
    y = 0 + y &= (-x + x) + y = -x + (x+y)\\
              &= -x + (x+z) = (-x + x) + z = 0 + z = z.
  \end{align*}
  This proves (a). Take $z = 0$ in (a) to obtain (b). Take $z = -x$ in (a) to obtain $(c)$. Since $-x + x = 0$, (c) with $-x$ in place of $x$ gives (d).
\end{proof}


\section*{Review}
\begin{enumerate}
  \item What do the axioms for addition imply? Explain.
\end{enumerate}


\section*{Links to Other Notes}
\begin{itemize}
  \item \hyperref[202501150657]{Definition of Field}
\end{itemize}

\section*{Table of Contents}

\begin{itemize}
  \item \hyperref[toc]{TOC}
\end{itemize}

