\clearpage
\phantomsection
\label{202501150809}
\renewcommand{\notetitle}{Implications of the Multiplication Axioms}

\section*{Note Information}
\begin{itemize}
  \item \textbf{ID:} \texttt{202501150809}
  \item \textbf{Timestamp:} \texttt{\today \ \currenttime}
  \item \textbf{Tags:} \texttt{Mathematics, Analysis-I, The-Real-and-Complex-Number-Systems}
  \item \textbf{References:}
    \begin{itemize}
      \item \href{https://ocw.mit.edu/courses/18-100b-analysis-i-fall-2010/}{Analysis I}
      \item \href{/home/garrett/Personal/References/Mathematics/Analysis-I/Rudin.pdf}{Rudin W., Principles of Mathematical Analysis}
    \end{itemize}
\end{itemize}


\section*{Main Content}
\textbf{Main Idea}\\
The axioms for multiplication imply the following statements.
\begin{itemize}
  \item[(a)] If $x \neq 0$ and $xy = xz$ then $y = z$.
  \item[(b)] If $x \neq 0$ and $xy = x$ then $y = 1$.
  \item[(c)] If $x \neq 0$ and $xy = 1$ then $y = 1/x$.
  \item[(d)] If $x \neq 0$ then $1/(1/x) = x$.\\
\end{itemize}

\textbf{Explanation}\\
\begin{proof}
  Suppose $x \neq 0$. Let $xy = xz$. The axioms (M) give
  \begin{align*}
    y = 1 \cdot y &= (x \cdot (1/x)) \cdot y = (1/x) \cdot (x \cdot y)\\
                  &= (1/x) \cdot (x \cdot z) = ((1/x) \cdot x) \cdot z = 1 \cdot z = z
  \end{align*}
  This proves (a). Take $z = 1$ in (a) to obtain (b). Take $z = (1/x)$ in (a) to obtain (c). Since $x(1/x) = 1$, (c) with $1/x$ in place of $y$ gives (d).\\
\end{proof}


\section*{Review}
\begin{enumerate}
  \item What do the axioms for multiplication imply? Explain.
\end{enumerate}


\section*{Links to Other Notes}
\begin{itemize}
  \item \hyperref[202501150657]{Definition of Field}
  \item \hyperref[202501150717]{Implications of Addition Axioms}
\end{itemize}

\section*{Table of Contents}

\begin{itemize}
  \item \hyperref[toc]{TOC}
\end{itemize}

