\clearpage
\phantomsection
\label{202501050700}
\renewcommand{\notetitle}{Motivation for Limits}

\section*{Note Information}
\begin{itemize}
  \item \textbf{ID:} \texttt{202501050700}
  \item \textbf{Timestamp:} \texttt{\today \ \currenttime}
  \item \textbf{Tags:} \texttt{Mathematics, Calculus-I, Limits}
  \item \textbf{References:}
    \begin{itemize}
      \item \href{https://ocw.mit.edu/courses/18-01-calculus-i-single-variable-calculus-fall-2020/}{Calculus I: Single Variable Calculus}
    \end{itemize}
\end{itemize}


\section*{Main Content}
\textbf{Main Idea}\\
Limits are foundational to the study of derivatives and integrals, the two main concepts in Calculus. 

\textbf{Explanation}\\
Imagine a curve, with two points labeled $A$ and $B$ (a). Let there be a line connecting the two points (b).
As $B$ approaches $A$ (c), the line becomes tangent to the curve at point $A$ (d). The slope of this tangent line represents the derivative at $A$.\\
\begin{center}
  \includegraphics[width=0.5\textwidth]{/home/garrett/Personal/Zettelkasten/3-Figures/IMG_1278.png}
\end{center}
Now, imagine a curvy region, and suppose we would like to measure its area (a). Let us start by filling the region with rectangles, since the area of rectangles is easy to measure (b).
As the width of each rectangle gets smaller, the total area of the rectangles gets closer to the area of the curvy region (c). Thus, the integral, which measures the area of a curvy region, 
is the limit of the total area of the rectangles as the width approaches 0.\\
\begin{center}
  \includegraphics[width=0.5\textwidth]{/home/garrett/Personal/Zettelkasten/3-Figures/IMG_1279.png}
\end{center}

\section*{Review}
\begin{enumerate}
  \item Describe the motivation behind studying limits.
\end{enumerate}

\section*{Links to Other Notes}
\begin{itemize}
  \item \hyperref[]{}
\end{itemize}

\section*{Table of Contents}
\begin{itemize}
  \item \hyperref[toc]{TOC}
\end{itemize}


