\clearpage
\phantomsection
\label{202501100655}
\renewcommand{\notetitle}{One-sided Limits}

\section*{Note Information}
\begin{itemize}
  \item \textbf{ID:} \texttt{202501100655}
  \item \textbf{Timestamp:} \texttt{\today \ \currenttime}
  \item \textbf{Tags:} \texttt{Mathematics, Calculus-I, Limits}
  \item \textbf{References:}
    \begin{itemize}
      \item \href{https://ocw.mit.edu/courses/18-01-calculus-i-single-variable-calculus-fall-2020/}{Calculus I: Single Variable Calculus}
    \end{itemize}
\end{itemize}


\section*{Main Content}
\textbf{Main Idea}\\
The left-sided limit is defined as the value $f(x)$ approaches as $x$ approaches $a$ from the left: 
\begin{align*}
  \lim_{x \rightarrow a^-} f(x).\\    
\end{align*}
The right-sided limit is defined as the value $f(x)$ approaches as $x$ approaches $a$ from the right:
\begin{align*}
  \lim_{x \rightarrow a^+} f(x).
\end{align*}

\textbf{Explanation}\\
Let $g(x) = \frac{x}{\tan(2x)}$. To find $\lim_{x \rightarrow 0^-} g(x)$, we must plug values slightly less than 0 into $g(x)$:
\begin{align*}
  g(-0.1) &= 0.493\\
  g(-0.01) &= 0.499\\
  g(-0.001) &= 0.500\\
\end{align*}
Thus, $\lim_{x \rightarrow 0^-} g(x) = 0.50$.

\section*{Review}
\begin{enumerate}
  \item Define the left-sided limit.
  \item Define the right-sided limit.
  \item Let $h(x) = \frac{|x| + \sin x}{x^2}$. Find $\lim_{x \rightarrow 0^+} h(x)$.
  \item Let $j(x) = \sin(13/x)$. Find $\lim_{x \rightarrow 0^+} j(x)$. 
\end{enumerate}


\section*{Links to Other Notes}
\begin{itemize}
  \item \hyperref[202501050700]{Motivation for Limits}
  \item \hyperref[202501070739]{Moving Closer and Closer}
\end{itemize}

\section*{Table of Contents}
\begin{itemize}
  \item \hyperref[toc]{TOC}
\end{itemize}

