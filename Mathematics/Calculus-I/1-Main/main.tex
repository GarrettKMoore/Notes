\documentclass[12pt]{article}  % Document class, 'article' is a good choice for beginner papers

% Basic packages for mathematical typesetting
\usepackage{amsmath}   % Essential package for mathematical symbols and environments
\usepackage{amssymb}   % Provides additional symbols like \mathbb{R}, \mathbb{N}, etc.
\usepackage{amsfonts}  % Fonts for mathematical symbols
\usepackage{graphicx}  % For including images or diagrams
\usepackage{hyperref}  % For clickable references and links
\usepackage{geometry}  % Set page margins easily
\usepackage{fancyhdr}  % Custom header/footer
\usepackage{enumerate} % For customizable lists (numbered, bullet points)
\usepackage{datetime}

% Page and margin setup
\geometry{top=1in, bottom=1in, left=1in, right=1in}  % Adjust the margins of the page
\setlength{\parskip}{1ex plus 0.5ex minus 0.5ex}  % Add space between paragraphs
\setlength{\parindent}{0pt}  % Disable paragraph indentation

% Custom commands for frequently used mathematical symbols
\newcommand{\R}{\mathbb{R}}  % Real numbers symbol
\newcommand{\N}{\mathbb{N}}  % Natural numbers symbol
\newcommand{\Z}{\mathbb{Z}}  % Integers symbol
\newcommand{\Q}{\mathbb{Q}}  % Rational numbers symbol
\newcommand{\C}{\mathbb{C}}  % Complex numbers symbol

% Theorem, Definition, and Proof environments
\newtheorem{theorem}{Theorem}[section]   % Theorem numbering by section
\newtheorem{definition}[theorem]{Definition}  % Definition with the same numbering as Theorem
\newtheorem{example}[theorem]{Example}      % Example environment
\newtheorem{remark}[theorem]{Remark}        % Remark environment

\newcommand{\notetitle}{}

% Header/footer setup
\pagestyle{fancy}
\fancyhf{}
\fancyhead[L]{Garrett Moore}
\fancyhead[C]{\notetitle}
\fancyhead[R]{\today}


\begin{document}

\label{toc}
\clearpage
\renewcommand{\notetitle}{Table of Contents}
\label{toc}
\begin{enumerate}

\item The-Axiom-of-Completeness
\begin{enumerate}
\item \hyperref[202501180703]{Initial Definition for R}
\item \hyperref[202501180727]{Axiom of Completeness}
\item \hyperref[202501180734]{Upper and Lower Bounds}
\item \hyperref[202501180743]{Supremum and Infimum}
\item \hyperref[202501181241]{Maximum and Minimum}
\item \hyperref[202501181257]{Q and the Axiom of Completeness}
\end{enumerate}
\end{enumerate}

\newpage


\clearpage
\phantomsection
\label{202501050700}
\renewcommand{\notetitle}{Motivation for Limits}

\section*{Note Information}
\begin{itemize}
  \item \textbf{ID:} \texttt{202501050700}
  \item \textbf{Timestamp:} \texttt{\today \ \currenttime}
  \item \textbf{Tags:} \texttt{Mathematics, Calculus-I, Limits}
  \item \textbf{References:}
    \begin{itemize}
      \item \href{https://ocw.mit.edu/courses/18-01-calculus-i-single-variable-calculus-fall-2020/}{Calculus I: Single Variable Calculus}
    \end{itemize}
\end{itemize}


\section*{Main Content}
\textbf{Main Idea}\\
Limits are foundational to the study of derivatives and integrals, the two main concepts in Calculus. 

\textbf{Explanation}\\
Imagine a curve, with two points labeled $A$ and $B$ (a). Let there be a line connecting the two points (b).
As $B$ approaches $A$ (c), the line becomes tangent to the curve at point $A$ (d). The slope of this tangent line represents the derivative at $A$.\\
\begin{center}
  \includegraphics[width=0.5\textwidth]{/home/garrett/Personal/Zettelkasten/3-Figures/IMG_1278.png}
\end{center}
Now, imagine a curvy region, and suppose we would like to measure its area (a). Let us start by filling the region with rectangles, since the area of rectangles is easy to measure (b).
As the width of each rectangle gets smaller, the total area of the rectangles gets closer to the area of the curvy region (c). Thus, the integral, which measures the area of a curvy region, 
is the limit of the total area of the rectangles as the width approaches 0.\\
\begin{center}
  \includegraphics[width=0.5\textwidth]{/home/garrett/Personal/Zettelkasten/3-Figures/IMG_1279.png}
\end{center}

\section*{Review}
\begin{enumerate}
  \item Describe the motivation behind studying limits.
\end{enumerate}

\section*{Links to Other Notes}
\begin{itemize}
  \item \hyperref[]{}
\end{itemize}

\section*{Table of Contents}
\begin{itemize}
  \item \hyperref[toc]{TOC}
\end{itemize}




\end{document}
