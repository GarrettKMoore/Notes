\clearpage
\phantomsection
\label{202501120904}
\renewcommand{\notetitle}{Question 1}

\section*{Note Information}
\begin{itemize}
  \item \textbf{ID:} \texttt{202501120904}
  \item \textbf{Timestamp:} \texttt{\today \ \currenttime}
  \item \textbf{Tags:} \texttt{Tutoring, Chhean, Session-1}
  \item \textbf{References:}
    \begin{itemize}
      \item \href{}{}
    \end{itemize}
\end{itemize}


\section*{Main Content}
\textbf{Main Idea}\\
Suppose a particle is moving on the $x$-axis in a simple harmonic motion. Its velocity, in meters per second, at time $t$, for $0 \leq t \leq 100$ seconds, is given by $v(t) = -\frac{5}{3} \sin(\frac{t}{3})$.
The total distance traveled by the particle in the time interval $0 \leq t \leq 21 \pi$ seconds is 70 meters.\\

\textbf{Explanation}\\
The velocity of the particle is modeled by $v(t) = -\frac{5}{3} \sin (\frac{t}{3})$. 
The total distance the particle travels in the time interval $0 \leq t \leq 21 \pi$ is equal to $\int_{0}^{21 \pi} | v(t) | dt$, where $| v(t) | = \frac{5}{3} | \sin (\frac{t}{3}) |$. 
Since $\sin(\frac{t}{3}) = 0$ when $t = 3 n \pi$ for all integers $n$, the velocity function maintains its sign throughout the interval $[3 n \pi, 3(n+1) \pi]$. 
The period for the velocity function is $6 \pi$, thus twice the aforementioned interval is equal to the full period. 
This relationship can be modeled through the following expressions:
\begin{align*}
  \frac{5}{3} \int_0^{6\pi} | \sin (\frac{t}{3}) | dt &= \frac{5}{3} \cdot 2 \int_0^{3\pi} \sin (\frac{t}{3}) dt\\
                                                      &= \frac{5}{3} \cdot 2[-3 \cos(\frac{t}{3})]_0^(3\pi)\\
                                                      &= \frac{5}{3} \cdot 2[6]\\
                                                      &= 20
\end{align*}
The interval from 0 to $21 \pi$ is equal to 3.5 periods. Therefore, the total distance traveled by the particle is equal to
\begin{align*}
  3 \cdot 20 + \frac{5}{3} \cdot 6 &= 70 \text{ meters}
\end{align*}

\section*{Review}
\begin{enumerate}
  \item 
\end{enumerate}


\section*{Links to Other Notes}
\begin{itemize}
  \item \hyperref[]{}
\end{itemize}

\section*{Table of Contents}

\begin{itemize}
  \item \hyperref[toc]{TOC}
\end{itemize}

